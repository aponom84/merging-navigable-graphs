\documentclass{article}
\usepackage[utf8]{inputenc}
\usepackage{url}

\title{Three Algorithms for Merging Navigable Graphs}
\author{}
\date{}

\usepackage{natbib}
\usepackage{graphicx}
\usepackage{algorithm}
\usepackage{algorithmicx}
\usepackage[noend]{algpseudocode}
\usepackage{amsmath}
\DeclareMathOperator*{\argminA}{arg\,min} % Jan Hlavacek
\DeclareMathOperator*{\argminB}{argmin}   % Jan Hlavacek
\usepackage{amsfonts}
\usepackage{amssymb}
\usepackage{nameref}
\begin{document}

\maketitle

\section {Search}


% \begin{algorithm}
% \caption{\textsc{Search}($G, q, L$)}\label{alg:search}
% \textbf{Require:} Graph $G = (V, E)$, query $q \in \mathbb{R}^d$, $L \in \mathbb{Z}$ \\
% \textbf{Ensure:} approximate nearest neighbor $v^* \in V$
% \begin{algorithmic}[1]
% \State $C \gets \textsc{InitializeCandidates}()$
% \While{True}
%     \State $u \gets$ nearest unvisited point to $q$ in $C$
%     \State $U \gets \{v \mid (u, v) \in E\}$
%     \For{$v \in U$}
%         \If{$v$ is not visited}
%             \State $C \gets C \cup \{v\}$
%         \EndIf
%     \EndFor
%     \If{$|C| > L$}
%         \State $C \gets$ top $L$ nearest points to $q$ in $C$
%     \EndIf
%     \If{$C$ is not updated}
%         \State \textbf{break}
%     \EndIf
% \EndWhile
% \State \Return nearest point to $q$ in $C$
% \end{algorithmic}
% \end{algorithm}

We start from the basic search over the graph. 

\begin{algorithm}
\caption{\textsc{LocalSearch}($G, q, C, k, L$)}\label{alg:local_search}
\textbf{Require:} Graph $G = (V, E)$, query $q \in \mathbb{R}^d$, initial set of candidate vertexes $C \subset V$,  $k \in \mathbb{N}$, $L \in \mathbb{N}$ \\
\textbf{Ensure:} approximate k-nearest neighbors $V^* \subset V$
\begin{algorithmic}[1]
% \State $C \gets \textsc{InitializeCandidates}()$
\While{True}
    \State $u \gets$ nearest unvisited point to $q$ in $C$
    \State $U \gets \{v \mid (u, v) \in E\}$
    \For{$v \in U$}
        \If{$v$ is not visited}
            \State $C \gets C \cup \{v\}$
        \EndIf
    \EndFor
    \If{$|C| > L$}
        \State $C \gets$ top $L$ nearest points to $q$ in $C$
    \EndIf
    \If{$C$ is not updated}
        \State \textbf{break}
    \EndIf
\EndWhile
\State \Return nearest point to $q$ in $C$
\end{algorithmic}
\end{algorithm}

$L$ is a parameter that controls the width of search. 
It helps to avoid local minima vertexes. In [HNSW] paper it is named \textbf{ef} parameter (expansion factor). 



\begin{algorithm}
\caption{\textsc{HNSW-Search}($\mathcal{H}, q, v_0, k, L, \ell $)}\label{alg:hnsw_search}
\textbf{Require:} The HNSW layers represented as  a sequence of graphs $\mathcal{H} = (G_i)_{i=0}^{l_{max}} )$, query $q \in \mathbb{R}^d$, initial vertex $v_0 \in V$,  $k \in \mathbb{N}$, $L \in \mathbb{N}$, the layer number in which the search should be performed $\ell$ \\    
\textbf{Ensure:} approximate k-nearest neighbors $V^* \subset V$
\begin{algorithmic}[1]
% \State $C \gets \textsc{InitializeCandidates}()$

\State $v^* \gets v_0$ 
\For{$i = l_{max} \; \textbf{to} \; \ell $}
    \State $v^* \gets \textsc{LocalSearch}(G=G_i, q=q, C={v^*}, k=1, L=L)$
\EndFor

\State \Return \textsc{LocalSearch}($ G=G_{\ell}, q=q, C={v^*}, k=k, L=L$)
\end{algorithmic}
\end{algorithm}

\section {Neighborhood Construction Strategies}

There are different strategies to form a vertex neighborhood. Our proposed merge algorithms are independent in neighborhood procedures. Merge algorithm can use a neighborhood construction procedure as an input parameter.

\begin{algorithm}
\caption{\textsc{KNN-Neighborhood-Construction}$(u,U, k)$}
\label{alg:knntrategy}
\textbf{Require:} vertex $u \in V$, neighbor candidates $U \subseteq V$, $k \in \mathbb{N} $\\
\textbf{Ensure:} selected neighbors $U' \subseteq U$
\begin{algorithmic}[1]
    
    \State $U' \gets \text{k-closest neighbors of } U \text{ to } u$
    \State \Return $U'$
\end{algorithmic}
\end{algorithm}

\begin{algorithm}
\caption{\textsc{RNG-Neighborhood-Construction}$(u, U, m)$}
\label{alg:rngstrategy}
\textbf{Require:} vertex $u \in V$, neighbor candidates $U \subseteq V$, maximum neighborhood size $m \in \mathbb{N}$\\
\textbf{Ensure:} selected neighbors $U' \subseteq U$
\begin{algorithmic}[1]
    \State sort $v \in U$ in ascending order of $\delta(u,v)$
    \State $U' \gets \emptyset$
    \For{$v \in U$}
        \State $f \gets \text{true}$
        \For{$w \in U'$}
            \If{$\delta(u,v) \geq \delta(v,w)$}
                \State $f \gets \text{false}$
                \State \textbf{break}
            \EndIf
        \EndFor
        \If{$f$}
            \State $U' \gets U' \cup \{v\}$
        \EndIf

        \If{$|U'| \geq m $}
            \State \textbf{break}
        \EndIf
    \EndFor
    \State \Return $U'$
\end{algorithmic}
\end{algorithm}

\section{Merge Algorithms}

We provide 3 merge algorithms: "Naive Merge", "Merge1", and "Merge2". We also refer to these algorithm as layer-merge algorithms. "Naive Merge" is a straight forward method, while "Merge1" and "Merge2" are more advanced. Together, these two algorithms constitute the main contribution of the work. The HNSW structure is a sequence of graphs, called layers. We start from description of a general algorithm for merging multi layers structure.

\subsection{HNSW General Merge}

HNSW-General-Merge algorithm \ref{alg:general_merge} is a straight forward. It uses one of the layer-merge algorithm to merge each HNSW layer. To able layer-merge algorithm to perform fast search using multi-layer structure of HNSW, HNSW-General-Merge algorithm passes to layer merge algorithm both HNSW structures $\mathcal{H}_a, \mathcal{H}_b$, and the layer number $\ell=i$ to which the merge should be performed. An order of merging layers doesn't mater. Thus merging of layers can be done in parallel.

\begin{algorithm}
\caption{\textsc{HNSW-General-Merge}($\mathcal{H}_a, \mathcal{H}_b$)}\label{alg:general_merge}
\textbf{Require:} The HNSW graphs $\mathcal{H}_a = (G^a_i), \mathcal{H}_b = (G^b_i)$ \\
\textbf{Ensure:}  Merged HNSW graphs $\mathcal{H}_c = (G^c_i)$ 
\begin{algorithmic}[1]

\For{$i = 0 \textbf{ to } \max(|\mathcal{H}_a|, |\mathcal{H}_b|) $}
    \State $G^c_i \gets \text{Merge}(\mathcal{H}_a, \mathcal{H}_b, \ell=i, \text{algorithm specific params})$
\EndFor

\State $\mathcal{H}_c \gets (G^c_i)_i^{\max(|\mathcal{H}_a|, |\mathcal{H}_b|)}$

\State \Return $\mathcal{H}^c$
\end{algorithmic}
\end{algorithm}

\subsection{Naive Merge}
The Naive Merge algorithm \ref{alg:merge_naive} surprisingly is very simple. It takes one vertex $v^*$ by one from the graph $G^a$. Then algorithm prepares the candidate set $\mathcal{C}$ for a new neighborhood (lines 7,8). It uses \textsc{HNSW-Search} to get $m$-closest vertices to $v^*$ in the graph $G^b$, and combines with an old neighborhood of the vertex $v^*$. After that in the line 9 the candidate set $\mathcal{C}$ is passed into the neighborhood construction function, which selects the vertices for the new merged neighborhood for $v^*$.
Lines 10-13 contains the same logic for the forming new neigborhoods of graph $G^b$

\begin{algorithm}
\caption{\textsc{Merge-Naive}($\mathcal{H}_a, \mathcal{H}_b,  \ell, \text{search\_ef}$)}\label{alg:merge_naive}
\textbf{Require:} The HNSW graphs $\mathcal{H}_a = (G^a_i)$, $\mathcal{H}_b = (G^b_i)$, and the level number $\ell$, search parameter $\text{search\_ef}$ \\
\textbf{Ensure:} Merged graph $G^c$ 
\begin{algorithmic}[1]
\State $V^c \gets V^a \cup V^b$
\State $E^c \gets \emptyset$
\State $G^a \gets G^a_{\ell}$
\State $G^b \gets G^b_{\ell}$
\State $m \gets \mathcal{H}_a.m_0$ if $\ell = 0$ else $\mathcal{H}_a.m$

\For{$v^* \in V^a$}
    \State $\mathcal{C}^b \gets \textsc{HNSW-Search}(\mathcal{H} = \mathcal{H}^b, q = v^*, v_0, k = m, L = \text{search\_ef}, \ell)$
    \State $\mathcal{C} \gets \{v : (v^*, v) \in E^a \} \cup \mathcal{C}^b$
    \State $E^c \gets E^c \cup \{(v^*, v) : v \in \textsc{neighborhood\_construction}(\mathcal{C}, v^*, m)\}$
\EndFor

\For{$v^* \in V^b$}
    \State $\mathcal{C}^a \gets \textsc{HNSW-Search}(\mathcal{H} = \mathcal{H}^a, q = v^*, v_0, k = m, L = \text{search\_ef}, \ell)$
    \State $\mathcal{C} \gets \{v : (v^*, v) \in E^b \} \cup \mathcal{C}^a$
    \State $E^c \gets E^c \cup \{(v^*, v) : v \in \textsc{neighborhood\_construction}(\mathcal{C}, v^*, m)\}$
\EndFor

\State \Return $G^c = (V^c, E^c)$
\end{algorithmic}
\end{algorithm}

\subsection{Merge1}
The most effort of \textsc{Merge-Naive} algorithm lies in the set of neighborhood candidates from the opposite graph utilising \textsc{HNSW-Search} procedure, that traverses layers graph from the top-level down to layer number $\ell$ every time. The number of computations can be reduced if we select the next vertex $v^*$ close to the previous one (line 15), instead of choosing it randomly. Thus, for new $v^*$ the neighborhood candidates will be  also close to the previous candidates set. To search this new neighborhood candidates we can use \textsc{LocalSearch} procedure, that traversing the same graph staring from the previous neighborhood candidates set $\mathcal{P}^b$ (line 11,14). In the line 14 in set $\mathcal{P}^b$ we keep only $M$-closest to $v^*$ candidates.\\
In the line 15 almost at each iteration we try select new $v^*$ vertex close to the previous  $v^*$ that was not already processed.  
To ensure that new $v^*$ vertex are not very far from the previous $v^*$ we bound the size of results of \textsc{LocalSearch} procedure controlling by "next\_step\_k" parameter. Once \textsc{LocalSearch} can't find enough close not processed vertex, we choice new $v^*$ from the set $\mathcal{V}_not\_done$ randomly.

\begin{algorithm}
\caption{\textsc{Merge1}($\mathcal{H}_a, \mathcal{H}_b, \ell, \text{jump\_ef}, \text{local\_ef}, \text{next\_step\_k}, M, m$)}\label{alg:merge1}
\textbf{Require:} The HNSW graphs $\mathcal{H}_a = (G^a_i), \mathcal{H}_b = (G^b_i)$, the merging layer number $\ell$, the size of the forming neighborhoods $m \in \mathbb{N}$, parameters $\text{jump\_ef}, \text{local\_ef}, \text{next\_step\_k} \in \mathbb{N}$ \\
\textbf{Ensure:}  Merged graph $G^c$ 
\begin{algorithmic}[1]
\State $V^c \gets V^a \cup V^b$
\State $E^c \gets \emptyset$
\State $G^a \gets G^a_{\ell} $
\State $G^b \gets G^b_{\ell} $
\State $\mathcal{V}_{not\_done} \gets V^a$

\While{$\mathcal{V}_{not\_done} \neq \emptyset$}
    \State $v^* \gets \text{random choice from } \mathcal{V}_{not\_done}$
    
    \State $\mathcal{P}^b  \gets \textsc{HNSW-Search}(\mathcal{H}=\mathcal{H}^b, q=v^*, v_0, k=M, L=\text{jump\_ef}, \ell)$
    
    \While{True}
        \State $\mathcal{V}_{not\_done} \gets \mathcal{V}_{not\_done} \setminus \{v^*\}$
        
        \State $\mathcal{C}^b  \gets \textsc{LocalSearch}(G=G^b, q=v^*, C=\mathcal{P}^b , k=m, L=\text{local\_ef})$
        
        \State $\mathcal{C} \gets  \{v : (v^*, v) \in E^a \} \cup \mathcal{C}^b$
        
        \State $E^c \gets E^c \cup  \{ (v^*, v)  : v \in \textsc{NeighborhoodConstruction}(\mathcal{C}, v^*, m) \}$

        \State $\mathcal{P}^b \gets \{\mathcal{C}^b_1, \mathcal{C}^b_2, ..., \mathcal{C}^b_M \} $

        \State $\mathcal{C}^a  \gets \textsc{LocalSearch}(G=G^a, q=v^*, C=\{v^*\} , k=\text{next\_step\_k}, L=\text{next\_step\_ef})$
        
        \State $\mathcal{C}^a \gets \mathcal{C}^a \cap \mathcal{V}_{not\_done}$
        
        \If{$\mathcal{C}^a = \emptyset$}
            \State \textbf{break}
        \EndIf
        
        \State $v^* \gets \mathcal{C}^a_1$
    \EndWhile
\EndWhile

\State $\mathcal{V}_{not\_done} \gets V^b$

\While{$\mathcal{V}_{not\_done} \neq \emptyset$}
    \State Repeat the same process for $V^b$ with the roles of $\mathcal{H}_a$ and $\mathcal{H}_b$ swapped.
\EndWhile

\State \Return $G^c=(V^c,E^c)$

\end{algorithmic}
\end{algorithm}



\subsection{Merge2}





\begin{algorithm}
\caption{\textsc{Merge2}($\mathcal{H}^a, \mathcal{H}^b,  \ell, \text{jump\_ef}, \text{local\_ef}, \text{next\_step\_k}, M, m$)}\label{alg:merge2}
\textbf{Require:} The HNSW graphs $\mathcal{H}_a = (G^a_i), \mathcal{H}_b = (G^b_i)$, the number of merged layers, the size of the forming neighborhoods $m \in \mathbb{N} $, parameters $\text{jump\_ef}, \text{local\_ef}, \text{next\_step\_k}, M \in \mathbb{N}$ \\
\textbf{Ensure:}  Merged graph $G^c$ 
\begin{algorithmic}[1]

\State $V^c \gets V^a \cup V^b$
\State $E^c \gets \emptyset$
\State $G^a \gets G^a_{\ell} $
\State $G^b \gets G^b_{\ell} $
% \State $m \gets \mathcal{H}_a.m_0 \text{ if } \ell = 0 \text{ else } \mathcal{H}_a.m$
\State $\mathcal{V}_{not\_done} \gets V^a \cup V^b$

\While{$\mathcal{V}_{not\_done} \neq \emptyset$}
    \State $v^* \gets \text{randomly choice from } \mathcal{V}_{not\_done}$

    \State $\mathcal{P}^a  \gets \textsc{HNSW-Search}(\mathcal{H}=\mathcal{H}^a, q=v^*, v_0, k, L=\text{jump\_ef}, \ell $)

    \State $\mathcal{P}^b \gets \textsc{HNSW-Search}(\mathcal{H}=\mathcal{H}^b, q=v^*, v_0, k, L=\text{jump\_ef}, \ell $)
    
    
    \While{True}
        \State $\mathcal{V}_{not\_done} \gets \mathcal{V}_{not\_done} \setminus \{v^*\}$
        

        \State $ \mathcal{C}^a  \gets \textsc{LocalSearch}(G=G^a, q=v^*, C=\mathcal{P}^a  , k=m, L=\text{local\_ef})$
        
        \State $\mathcal{C}^b \gets \textsc{LocalSearch}(G=G^b, q=v^*, C=\mathcal{P}^b , k=m, L=\text{local\_ef})$
        
        \If{$v^* \in V^a $}
            \State $\mathcal{C}\gets  \{v : (v^*, v) \in E^a \} \cup  \mathcal{C}^b\}$
        \Else
            \State $\mathcal{C} \gets  \{v : (v^*, v) \in E^b \} \cup  \mathcal{C}^a\}$
        \EndIf
        
        \State $E^c \gets E^c \cup \{ (v^*, v) : v \in \text{neighborhood\_construction}(\mathcal{C}, v^*, m) \}$
        
        \State $\mathcal{C}^a_{not\_done} \gets \{\mathcal{C}^a_1, \mathcal{C}^a_2, ..., \mathcal{C}^a_{ \text{next\_step\_k} } \} \cap \mathcal{V}_{not\_done}$

        \State $\mathcal{C}^b_{not\_done} \gets \{\mathcal{C}^b_1, \mathcal{C}^b_2, ..., \mathcal{C}^b_{ \text{next\_step\_k} } \}  \cap \mathcal{V}_{not\_done}$
        
        
        \State $\mathcal{C}_{not\_done} \gets \mathcal{C}^a_{not\_done} \cup \mathcal{C}^b_{not\_done}$
        
        \If{$\mathcal{C}_{not\_done} = \emptyset$}
            \State \textbf{break}
        \EndIf
        
        \State $v^* \gets \underset{v \in \mathcal{C}_{not\_done}}{\mathrm{argmin}} \rho(v, v^*)$
        
        \State $\mathcal{P}_a \gets \mathcal{C}^a$
        \State $\mathcal{P}_b \gets \mathcal{C}^b$
    \EndWhile
\EndWhile

\State \Return $G^c=(V^c,E^c)$


\end{algorithmic}
\end{algorithm}

\section{Computational Experiments}
TODO




\bibliographystyle{plain}
\bibliography{references}
\end{document}
